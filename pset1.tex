\documentclass{article}

\usepackage{amsmath}
\usepackage{color}
\usepackage{listings}
\usepackage{xfrac}

\definecolor{dkgreen}{rgb}{0,0.6,0}
\definecolor{gray}{rgb}{0.5,0.5,0.5}
\definecolor{mauve}{rgb}{0.58,0,0.82}

\lstset{ %
  language=Python,                % the language of the code
  basicstyle=\footnotesize,           % the size of the fonts that are used for the code
  numbers=left,                   % where to put the line-numbers
  numberstyle=\tiny\color{gray},  % the style that is used for the line-numbers
  stepnumber=2,                   % the step between two line-numbers. If it's 1, each line 
                                  % will be numbered
  numbersep=5pt,                  % how far the line-numbers are from the code
  backgroundcolor=\color{white},      % choose the background color. You must add \usepackage{color}
  showspaces=false,               % show spaces adding particular underscores
  showstringspaces=false,         % underline spaces within strings
  showtabs=false,                 % show tabs within strings adding particular underscores
  frame=single,                   % adds a frame around the code
  rulecolor=\color{black},        % if not set, the frame-color may be changed on line-breaks within not-black text (e.g. commens (green here))
  tabsize=2,                      % sets default tabsize to 2 spaces
  captionpos=b,                   % sets the caption-position to bottom
  breaklines=true,                % sets automatic line breaking
  breakatwhitespace=true,        % sets if automatic breaks should only happen at whitespace
  title=\lstname,                   % show the filename of files included with \lstinputlisting;
                                  % also try caption instead of title
  keywordstyle=\color{blue},          % keyword style
  commentstyle=\color{dkgreen},       % comment style
  stringstyle=\color{mauve},         % string literal style
  escapeinside={\%*}{*)},            % if you want to add a comment within your code
  morekeywords={*,...}               % if you want to add more keywords to the set
}

\title{Problem Set 1}
\author{Alec Story \\ \small{avs38}}

\begin{document}
\maketitle

\section{}

    If $Y = a + bX$, $E[X] = \mu $ and $Var(X) = \sigma^2$, then 

    \begin{align*}
        E[Y] &= E[a + bX] \\
             &= a + b \cdot E[X] \\
             &= a + b\mu
    \end{align*}

    \begin{align*}
        Var[Y] &= Var[a + bX] \\
               &= Var[bX] \\
               &= b^2 \cdot Var[X] \\
               &= b^2 \sigma^2
    \end{align*}

\section{}

A heads is more likely to be flipped before a 2 is rolled:  the chance of
getting a head is $\sfrac{1}{2}$, which is greater than $\sfrac{1}{6}$, the
chance of rolling a 2.

The number of times we flip a coin before a 6 is rolled is geometrically
distributed with parameter $p = 1/6$.  Therefore, the expected number of times
we flip a coin is $\frac{1}{p} = 6$, and since we expect half of those trials to
be tails, the expected number of tails before a 6 is rolled is 3.

\section{}

    \subsection*{X}

    First, we find the distribution of X, the minimum of n independent
    exponential random variables $E_1 \dots E_n$:

    \begin{align*}
            F_X(t) &= P(\text{min}(E_1, \dots, E_n) \le t) \\
        1 - F_X(t) &= P(\text{min}(E_1, \dots, E_n) > t) \\
                   &= P(E_1 > t \land \dots \land E_n > t) \\
                   &= \prod_{i=1}^n P(E_i > t) \\
                   &= \prod_{i=1}^n 1 - F_i(t) \\
                   &= \prod_{i=1}^n e^{-\mu_i t} \\
                   &= e^{-(\sum_{i=1}^n\mu_i) t} \\
            F_X(t) &= 1 - e^{-(\sum_{i=1}^n\mu_i) t} \\
            f_X(t) &= (\sum_{i=1}^n\mu_i) e^{-(\sum_{i=1}^n\mu_i) t}
    \end{align*}

    Which is the same as an exponential random variable with $\mu =
    \sum_{i=1}^n\mu_i$

    \subsection*{Y}

    Next, we find the distribution of Y, the maximum of n independent
    exponential random variables $E_1 \dots E_n$,

    \begin{align*}
            F_X(t) &= P(\text{max}(E_1, \dots, E_n) \le t) \\
                   &= P(E_1 \le t \land \dots \land E_n \le t) \\
                   &= \prod_{i=1}^n P(E_i \le t) \\
                   &= \prod_{i=1}^n F_i(t) \\
                   &= \prod_{i=1}^n (1 - e^{-\mu_i t})
    \end{align*}

    Which fully describes the distribution, but is not differentiable without a
    fixed n.  However, if the exponential random variables are identically
    distributed with parameter $\mu'$, we get additionally,
    \begin{align*}
            F_X(t) &= \prod_{i=1}^n (1 - e^{-\mu' t}) \\
                   &= (1 - e^{-\mu' t})^n \\
            f_X(t) &= n (1 - e^{-\mu' t})^{n-1} \cdot \mu' e^{-\mu' t}
    \end{align*}


\section{}

    If both parents are heterozygous, the probability of a healthy child being
    homozygous for the normal allele is $\sfrac{1}{3}$.  Of the four possible
    outcomes from the parents, one is removed because we know the child is
    healthy, two result in a heterozygous child, and one results in a homozygous
    normal child.

    If one parent is homozygous and the other is heterozygous, then the child
    will always inherit one normal allele from its homozygous normal parent, so
    the chance of being homozygous normal is the chance of inheriting the normal
    allele from the heterozygous parent, and is $\sfrac{1}{2}$.

\section{}

\begin{figure}[h]
\begin{center}
    % GNUPLOT: LaTeX picture
\setlength{\unitlength}{0.240900pt}
\ifx\plotpoint\undefined\newsavebox{\plotpoint}\fi
\begin{picture}(1500,900)(0,0)
\sbox{\plotpoint}{\rule[-0.200pt]{0.400pt}{0.400pt}}%
\put(171.0,131.0){\rule[-0.200pt]{4.818pt}{0.400pt}}
\put(151,131){\makebox(0,0)[r]{ 0}}
\put(1419.0,131.0){\rule[-0.200pt]{4.818pt}{0.400pt}}
\put(171.0,260.0){\rule[-0.200pt]{4.818pt}{0.400pt}}
\put(151,260){\makebox(0,0)[r]{ 0.1}}
\put(1419.0,260.0){\rule[-0.200pt]{4.818pt}{0.400pt}}
\put(171.0,389.0){\rule[-0.200pt]{4.818pt}{0.400pt}}
\put(151,389){\makebox(0,0)[r]{ 0.2}}
\put(1419.0,389.0){\rule[-0.200pt]{4.818pt}{0.400pt}}
\put(171.0,518.0){\rule[-0.200pt]{4.818pt}{0.400pt}}
\put(151,518){\makebox(0,0)[r]{ 0.3}}
\put(1419.0,518.0){\rule[-0.200pt]{4.818pt}{0.400pt}}
\put(171.0,647.0){\rule[-0.200pt]{4.818pt}{0.400pt}}
\put(151,647){\makebox(0,0)[r]{ 0.4}}
\put(1419.0,647.0){\rule[-0.200pt]{4.818pt}{0.400pt}}
\put(171.0,776.0){\rule[-0.200pt]{4.818pt}{0.400pt}}
\put(151,776){\makebox(0,0)[r]{ 0.5}}
\put(1419.0,776.0){\rule[-0.200pt]{4.818pt}{0.400pt}}
\put(171.0,131.0){\rule[-0.200pt]{0.400pt}{4.818pt}}
\put(171,90){\makebox(0,0){ 0}}
\put(171.0,756.0){\rule[-0.200pt]{0.400pt}{4.818pt}}
\put(488.0,131.0){\rule[-0.200pt]{0.400pt}{4.818pt}}
\put(488,90){\makebox(0,0){ 5}}
\put(488.0,756.0){\rule[-0.200pt]{0.400pt}{4.818pt}}
\put(805.0,131.0){\rule[-0.200pt]{0.400pt}{4.818pt}}
\put(805,90){\makebox(0,0){ 10}}
\put(805.0,756.0){\rule[-0.200pt]{0.400pt}{4.818pt}}
\put(1122.0,131.0){\rule[-0.200pt]{0.400pt}{4.818pt}}
\put(1122,90){\makebox(0,0){ 15}}
\put(1122.0,756.0){\rule[-0.200pt]{0.400pt}{4.818pt}}
\put(1439.0,131.0){\rule[-0.200pt]{0.400pt}{4.818pt}}
\put(1439,90){\makebox(0,0){ 20}}
\put(1439.0,756.0){\rule[-0.200pt]{0.400pt}{4.818pt}}
\put(171.0,131.0){\rule[-0.200pt]{0.400pt}{155.380pt}}
\put(171.0,131.0){\rule[-0.200pt]{305.461pt}{0.400pt}}
\put(1439.0,131.0){\rule[-0.200pt]{0.400pt}{155.380pt}}
\put(171.0,776.0){\rule[-0.200pt]{305.461pt}{0.400pt}}
\put(30,453){\makebox(0,0){Heterozygosity}}
\put(805,29){\makebox(0,0){Generations}}
\put(805,838){\makebox(0,0){5 Runs at 10 Individuals}}
\put(171,750){\usebox{\plotpoint}}
\multiput(171.58,747.56)(0.499,-0.611){123}{\rule{0.120pt}{0.589pt}}
\multiput(170.17,748.78)(63.000,-75.778){2}{\rule{0.400pt}{0.294pt}}
\multiput(234.58,673.00)(0.499,0.806){125}{\rule{0.120pt}{0.744pt}}
\multiput(233.17,673.00)(64.000,101.456){2}{\rule{0.400pt}{0.372pt}}
\multiput(298.58,772.87)(0.499,-0.819){123}{\rule{0.120pt}{0.754pt}}
\multiput(297.17,774.44)(63.000,-101.435){2}{\rule{0.400pt}{0.377pt}}
\multiput(361.58,664.54)(0.499,-2.432){125}{\rule{0.120pt}{2.038pt}}
\multiput(360.17,668.77)(64.000,-305.771){2}{\rule{0.400pt}{1.019pt}}
\multiput(425.58,356.47)(0.499,-1.848){123}{\rule{0.120pt}{1.573pt}}
\multiput(424.17,359.74)(63.000,-228.735){2}{\rule{0.400pt}{0.787pt}}
\put(488.0,131.0){\rule[-0.200pt]{229.096pt}{0.400pt}}
\put(171,750){\usebox{\plotpoint}}
\multiput(171,750)(13.143,-16.064){5}{\usebox{\plotpoint}}
\multiput(234,673)(20.756,0.000){3}{\usebox{\plotpoint}}
\multiput(298,673)(13.143,16.064){5}{\usebox{\plotpoint}}
\multiput(361,750)(6.158,-19.821){11}{\usebox{\plotpoint}}
\multiput(425,544)(20.756,0.000){3}{\usebox{\plotpoint}}
\multiput(488,544)(6.823,-19.602){9}{\usebox{\plotpoint}}
\multiput(551,363)(20.756,0.000){3}{\usebox{\plotpoint}}
\multiput(615,363)(5.439,-20.030){11}{\usebox{\plotpoint}}
\multiput(678,131)(20.756,0.000){4}{\usebox{\plotpoint}}
\multiput(742,131)(20.756,0.000){3}{\usebox{\plotpoint}}
\multiput(805,131)(20.756,0.000){3}{\usebox{\plotpoint}}
\multiput(868,131)(20.756,0.000){3}{\usebox{\plotpoint}}
\multiput(932,131)(20.756,0.000){3}{\usebox{\plotpoint}}
\multiput(995,131)(20.756,0.000){3}{\usebox{\plotpoint}}
\multiput(1059,131)(20.756,0.000){3}{\usebox{\plotpoint}}
\multiput(1122,131)(20.756,0.000){3}{\usebox{\plotpoint}}
\multiput(1185,131)(20.756,0.000){3}{\usebox{\plotpoint}}
\multiput(1249,131)(20.756,0.000){3}{\usebox{\plotpoint}}
\multiput(1312,131)(20.756,0.000){3}{\usebox{\plotpoint}}
\multiput(1376,131)(20.756,0.000){3}{\usebox{\plotpoint}}
\put(1439,131){\usebox{\plotpoint}}
\sbox{\plotpoint}{\rule[-0.400pt]{0.800pt}{0.800pt}}%
\put(171,750){\usebox{\plotpoint}}
\multiput(234.00,751.41)(1.250,0.504){45}{\rule{2.169pt}{0.121pt}}
\multiput(234.00,748.34)(59.498,26.000){2}{\rule{1.085pt}{0.800pt}}
\multiput(298.00,774.09)(1.230,-0.504){45}{\rule{2.138pt}{0.121pt}}
\multiput(298.00,774.34)(58.562,-26.000){2}{\rule{1.069pt}{0.800pt}}
\multiput(362.41,738.48)(0.501,-1.621){121}{\rule{0.121pt}{2.775pt}}
\multiput(359.34,744.24)(64.000,-200.240){2}{\rule{0.800pt}{1.388pt}}
\multiput(426.41,533.63)(0.502,-1.446){119}{\rule{0.121pt}{2.498pt}}
\multiput(423.34,538.81)(63.000,-175.814){2}{\rule{0.800pt}{1.249pt}}
\put(171.0,750.0){\rule[-0.400pt]{15.177pt}{0.800pt}}
\multiput(552.41,350.13)(0.501,-1.826){121}{\rule{0.121pt}{3.100pt}}
\multiput(549.34,356.57)(64.000,-225.566){2}{\rule{0.800pt}{1.550pt}}
\put(488.0,363.0){\rule[-0.400pt]{15.177pt}{0.800pt}}
\put(615.0,131.0){\rule[-0.400pt]{198.502pt}{0.800pt}}
\sbox{\plotpoint}{\rule[-0.500pt]{1.000pt}{1.000pt}}%
\put(171,750){\usebox{\plotpoint}}
\multiput(171,750)(6.070,-19.848){11}{\usebox{\plotpoint}}
\multiput(234,544)(6.919,-19.568){9}{\usebox{\plotpoint}}
\multiput(298,363)(5.439,-20.030){12}{\usebox{\plotpoint}}
\multiput(361,131)(20.756,0.000){3}{\usebox{\plotpoint}}
\multiput(425,131)(20.756,0.000){3}{\usebox{\plotpoint}}
\multiput(488,131)(20.756,0.000){3}{\usebox{\plotpoint}}
\multiput(551,131)(20.756,0.000){3}{\usebox{\plotpoint}}
\multiput(615,131)(20.756,0.000){3}{\usebox{\plotpoint}}
\multiput(678,131)(20.756,0.000){3}{\usebox{\plotpoint}}
\multiput(742,131)(20.756,0.000){3}{\usebox{\plotpoint}}
\multiput(805,131)(20.756,0.000){3}{\usebox{\plotpoint}}
\multiput(868,131)(20.756,0.000){3}{\usebox{\plotpoint}}
\multiput(932,131)(20.756,0.000){3}{\usebox{\plotpoint}}
\multiput(995,131)(20.756,0.000){3}{\usebox{\plotpoint}}
\multiput(1059,131)(20.756,0.000){3}{\usebox{\plotpoint}}
\multiput(1122,131)(20.756,0.000){3}{\usebox{\plotpoint}}
\multiput(1185,131)(20.756,0.000){3}{\usebox{\plotpoint}}
\multiput(1249,131)(20.756,0.000){4}{\usebox{\plotpoint}}
\multiput(1312,131)(20.756,0.000){3}{\usebox{\plotpoint}}
\multiput(1376,131)(20.756,0.000){3}{\usebox{\plotpoint}}
\put(1439,131){\usebox{\plotpoint}}
\sbox{\plotpoint}{\rule[-0.600pt]{1.200pt}{1.200pt}}%
\put(171,750){\usebox{\plotpoint}}
\multiput(298.00,752.24)(1.212,0.500){42}{\rule{3.208pt}{0.121pt}}
\multiput(298.00,747.51)(56.342,26.000){2}{\rule{1.604pt}{1.200pt}}
\multiput(363.24,766.74)(0.500,-0.801){118}{\rule{0.120pt}{2.231pt}}
\multiput(358.51,771.37)(64.000,-98.369){2}{\rule{1.200pt}{1.116pt}}
\multiput(427.24,661.55)(0.500,-1.022){116}{\rule{0.120pt}{2.757pt}}
\multiput(422.51,667.28)(63.000,-123.277){2}{\rule{1.200pt}{1.379pt}}
\put(171.0,750.0){\rule[-0.600pt]{30.594pt}{1.200pt}}
\multiput(553.24,544.00)(0.500,1.006){118}{\rule{0.120pt}{2.719pt}}
\multiput(548.51,544.00)(64.000,123.357){2}{\rule{1.200pt}{1.359pt}}
\put(488.0,544.0){\rule[-0.600pt]{15.177pt}{1.200pt}}
\multiput(680.24,673.00)(0.500,0.596){118}{\rule{0.120pt}{1.744pt}}
\multiput(675.51,673.00)(64.000,73.381){2}{\rule{1.200pt}{0.872pt}}
\multiput(742.00,752.24)(1.212,0.500){42}{\rule{3.208pt}{0.121pt}}
\multiput(742.00,747.51)(56.342,26.000){2}{\rule{1.604pt}{1.200pt}}
\multiput(807.24,756.41)(0.500,-1.847){116}{\rule{0.120pt}{4.719pt}}
\multiput(802.51,766.21)(63.000,-222.205){2}{\rule{1.200pt}{2.360pt}}
\multiput(870.24,544.00)(0.500,1.818){118}{\rule{0.120pt}{4.650pt}}
\multiput(865.51,544.00)(64.000,222.349){2}{\rule{1.200pt}{2.325pt}}
\multiput(932.00,773.26)(1.212,-0.500){42}{\rule{3.208pt}{0.121pt}}
\multiput(932.00,773.51)(56.342,-26.000){2}{\rule{1.604pt}{1.200pt}}
\multiput(997.24,732.72)(0.500,-1.613){118}{\rule{0.120pt}{4.163pt}}
\multiput(992.51,741.36)(64.000,-197.361){2}{\rule{1.200pt}{2.081pt}}
\put(615.0,673.0){\rule[-0.600pt]{15.177pt}{1.200pt}}
\multiput(1124.24,528.44)(0.500,-1.439){116}{\rule{0.120pt}{3.748pt}}
\multiput(1119.51,536.22)(63.000,-173.222){2}{\rule{1.200pt}{1.874pt}}
\put(1059.0,544.0){\rule[-0.600pt]{15.177pt}{1.200pt}}
\multiput(1378.24,363.00)(0.500,1.439){116}{\rule{0.120pt}{3.748pt}}
\multiput(1373.51,363.00)(63.000,173.222){2}{\rule{1.200pt}{1.874pt}}
\put(1185.0,363.0){\rule[-0.600pt]{46.012pt}{1.200pt}}
\sbox{\plotpoint}{\rule[-0.200pt]{0.400pt}{0.400pt}}%
\put(171.0,131.0){\rule[-0.200pt]{0.400pt}{155.380pt}}
\put(171.0,131.0){\rule[-0.200pt]{305.461pt}{0.400pt}}
\put(1439.0,131.0){\rule[-0.200pt]{0.400pt}{155.380pt}}
\put(171.0,776.0){\rule[-0.200pt]{305.461pt}{0.400pt}}
\end{picture}

\end{center}
\end{figure}

The observed heterozygosity after 20 generations is usually 0, but occasionally
non-zero, as this graph demonstrates.  This is expected, since under this model
heterozygosity should decline exponentially, on average, and since the
population is small, it should decline rapidly.

\begin{figure}[h]
\begin{center}
    % GNUPLOT: LaTeX picture
\setlength{\unitlength}{0.240900pt}
\ifx\plotpoint\undefined\newsavebox{\plotpoint}\fi
\begin{picture}(1500,900)(0,0)
\sbox{\plotpoint}{\rule[-0.200pt]{0.400pt}{0.400pt}}%
\put(191.0,131.0){\rule[-0.200pt]{4.818pt}{0.400pt}}
\put(171,131){\makebox(0,0)[r]{ 0.34}}
\put(1419.0,131.0){\rule[-0.200pt]{4.818pt}{0.400pt}}
\put(191.0,212.0){\rule[-0.200pt]{4.818pt}{0.400pt}}
\put(171,212){\makebox(0,0)[r]{ 0.36}}
\put(1419.0,212.0){\rule[-0.200pt]{4.818pt}{0.400pt}}
\put(191.0,292.0){\rule[-0.200pt]{4.818pt}{0.400pt}}
\put(171,292){\makebox(0,0)[r]{ 0.38}}
\put(1419.0,292.0){\rule[-0.200pt]{4.818pt}{0.400pt}}
\put(191.0,373.0){\rule[-0.200pt]{4.818pt}{0.400pt}}
\put(171,373){\makebox(0,0)[r]{ 0.4}}
\put(1419.0,373.0){\rule[-0.200pt]{4.818pt}{0.400pt}}
\put(191.0,454.0){\rule[-0.200pt]{4.818pt}{0.400pt}}
\put(171,454){\makebox(0,0)[r]{ 0.42}}
\put(1419.0,454.0){\rule[-0.200pt]{4.818pt}{0.400pt}}
\put(191.0,534.0){\rule[-0.200pt]{4.818pt}{0.400pt}}
\put(171,534){\makebox(0,0)[r]{ 0.44}}
\put(1419.0,534.0){\rule[-0.200pt]{4.818pt}{0.400pt}}
\put(191.0,615.0){\rule[-0.200pt]{4.818pt}{0.400pt}}
\put(171,615){\makebox(0,0)[r]{ 0.46}}
\put(1419.0,615.0){\rule[-0.200pt]{4.818pt}{0.400pt}}
\put(191.0,695.0){\rule[-0.200pt]{4.818pt}{0.400pt}}
\put(171,695){\makebox(0,0)[r]{ 0.48}}
\put(1419.0,695.0){\rule[-0.200pt]{4.818pt}{0.400pt}}
\put(191.0,776.0){\rule[-0.200pt]{4.818pt}{0.400pt}}
\put(171,776){\makebox(0,0)[r]{ 0.5}}
\put(1419.0,776.0){\rule[-0.200pt]{4.818pt}{0.400pt}}
\put(191.0,131.0){\rule[-0.200pt]{0.400pt}{4.818pt}}
\put(191,90){\makebox(0,0){ 0}}
\put(191.0,756.0){\rule[-0.200pt]{0.400pt}{4.818pt}}
\put(503.0,131.0){\rule[-0.200pt]{0.400pt}{4.818pt}}
\put(503,90){\makebox(0,0){ 5}}
\put(503.0,756.0){\rule[-0.200pt]{0.400pt}{4.818pt}}
\put(815.0,131.0){\rule[-0.200pt]{0.400pt}{4.818pt}}
\put(815,90){\makebox(0,0){ 10}}
\put(815.0,756.0){\rule[-0.200pt]{0.400pt}{4.818pt}}
\put(1127.0,131.0){\rule[-0.200pt]{0.400pt}{4.818pt}}
\put(1127,90){\makebox(0,0){ 15}}
\put(1127.0,756.0){\rule[-0.200pt]{0.400pt}{4.818pt}}
\put(1439.0,131.0){\rule[-0.200pt]{0.400pt}{4.818pt}}
\put(1439,90){\makebox(0,0){ 20}}
\put(1439.0,756.0){\rule[-0.200pt]{0.400pt}{4.818pt}}
\put(191.0,131.0){\rule[-0.200pt]{0.400pt}{155.380pt}}
\put(191.0,131.0){\rule[-0.200pt]{300.643pt}{0.400pt}}
\put(1439.0,131.0){\rule[-0.200pt]{0.400pt}{155.380pt}}
\put(191.0,776.0){\rule[-0.200pt]{300.643pt}{0.400pt}}
\put(30,453){\makebox(0,0){Heterozygosity}}
\put(815,29){\makebox(0,0){Generations}}
\put(815,838){\makebox(0,0){5 Runs at 1000 Individuals}}
\put(191,457){\usebox{\plotpoint}}
\multiput(191.00,455.92)(0.508,-0.499){119}{\rule{0.507pt}{0.120pt}}
\multiput(191.00,456.17)(60.949,-61.000){2}{\rule{0.253pt}{0.400pt}}
\multiput(253.58,392.63)(0.499,-0.890){123}{\rule{0.120pt}{0.811pt}}
\multiput(252.17,394.32)(63.000,-110.316){2}{\rule{0.400pt}{0.406pt}}
\multiput(316.00,282.92)(0.864,-0.498){69}{\rule{0.789pt}{0.120pt}}
\multiput(316.00,283.17)(60.363,-36.000){2}{\rule{0.394pt}{0.400pt}}
\multiput(378.00,246.92)(1.055,-0.497){57}{\rule{0.940pt}{0.120pt}}
\multiput(378.00,247.17)(61.049,-30.000){2}{\rule{0.470pt}{0.400pt}}
\multiput(441.58,215.36)(0.499,-0.670){121}{\rule{0.120pt}{0.635pt}}
\multiput(440.17,216.68)(62.000,-81.681){2}{\rule{0.400pt}{0.318pt}}
\multiput(503.58,135.00)(0.499,0.978){121}{\rule{0.120pt}{0.881pt}}
\multiput(502.17,135.00)(62.000,119.172){2}{\rule{0.400pt}{0.440pt}}
\multiput(565.00,254.92)(1.881,-0.495){31}{\rule{1.582pt}{0.119pt}}
\multiput(565.00,255.17)(59.716,-17.000){2}{\rule{0.791pt}{0.400pt}}
\multiput(628.00,239.58)(0.585,0.498){103}{\rule{0.568pt}{0.120pt}}
\multiput(628.00,238.17)(60.821,53.000){2}{\rule{0.284pt}{0.400pt}}
\multiput(690.00,292.58)(1.174,0.497){51}{\rule{1.033pt}{0.120pt}}
\multiput(690.00,291.17)(60.855,27.000){2}{\rule{0.517pt}{0.400pt}}
\multiput(753.00,319.58)(1.155,0.497){51}{\rule{1.019pt}{0.120pt}}
\multiput(753.00,318.17)(59.886,27.000){2}{\rule{0.509pt}{0.400pt}}
\multiput(815.00,344.92)(0.499,-0.499){121}{\rule{0.500pt}{0.120pt}}
\multiput(815.00,345.17)(60.962,-62.000){2}{\rule{0.250pt}{0.400pt}}
\multiput(877.00,282.92)(1.132,-0.497){53}{\rule{1.000pt}{0.120pt}}
\multiput(877.00,283.17)(60.924,-28.000){2}{\rule{0.500pt}{0.400pt}}
\multiput(940.00,254.92)(2.439,-0.493){23}{\rule{2.008pt}{0.119pt}}
\multiput(940.00,255.17)(57.833,-13.000){2}{\rule{1.004pt}{0.400pt}}
\multiput(1002.00,243.58)(0.958,0.497){63}{\rule{0.864pt}{0.120pt}}
\multiput(1002.00,242.17)(61.207,33.000){2}{\rule{0.432pt}{0.400pt}}
\multiput(1065.00,276.58)(1.970,0.494){29}{\rule{1.650pt}{0.119pt}}
\multiput(1065.00,275.17)(58.575,16.000){2}{\rule{0.825pt}{0.400pt}}
\multiput(1127.58,289.60)(0.499,-0.597){121}{\rule{0.120pt}{0.577pt}}
\multiput(1126.17,290.80)(62.000,-72.802){2}{\rule{0.400pt}{0.289pt}}
\multiput(1189.00,218.58)(1.270,0.497){47}{\rule{1.108pt}{0.120pt}}
\multiput(1189.00,217.17)(60.700,25.000){2}{\rule{0.554pt}{0.400pt}}
\multiput(1252.00,243.58)(0.690,0.498){87}{\rule{0.651pt}{0.120pt}}
\multiput(1252.00,242.17)(60.649,45.000){2}{\rule{0.326pt}{0.400pt}}
\multiput(1314.00,286.94)(9.108,-0.468){5}{\rule{6.400pt}{0.113pt}}
\multiput(1314.00,287.17)(49.716,-4.000){2}{\rule{3.200pt}{0.400pt}}
\multiput(1377.00,284.58)(1.568,0.496){37}{\rule{1.340pt}{0.119pt}}
\multiput(1377.00,283.17)(59.219,20.000){2}{\rule{0.670pt}{0.400pt}}
\put(191,457){\usebox{\plotpoint}}
\multiput(191,457)(19.141,-8.027){4}{\usebox{\plotpoint}}
\multiput(253,431)(15.513,-13.789){4}{\usebox{\plotpoint}}
\multiput(316,375)(17.823,-10.636){3}{\usebox{\plotpoint}}
\multiput(378,338)(20.756,0.000){3}{\usebox{\plotpoint}}
\multiput(441,338)(19.460,7.219){4}{\usebox{\plotpoint}}
\multiput(503,361)(20.246,4.572){3}{\usebox{\plotpoint}}
\multiput(565,375)(18.386,-9.631){3}{\usebox{\plotpoint}}
\multiput(628,342)(16.029,13.185){4}{\usebox{\plotpoint}}
\multiput(690,393)(18.623,9.164){3}{\usebox{\plotpoint}}
\multiput(753,424)(16.284,-12.870){4}{\usebox{\plotpoint}}
\multiput(815,375)(20.624,-2.329){3}{\usebox{\plotpoint}}
\multiput(877,368)(19.690,6.563){3}{\usebox{\plotpoint}}
\multiput(940,389)(16.029,-13.185){4}{\usebox{\plotpoint}}
\multiput(1002,338)(19.497,7.118){3}{\usebox{\plotpoint}}
\multiput(1065,361)(17.184,11.641){4}{\usebox{\plotpoint}}
\multiput(1127,403)(18.916,-8.543){3}{\usebox{\plotpoint}}
\multiput(1189,375)(18.854,-8.679){4}{\usebox{\plotpoint}}
\multiput(1252,346)(20.436,3.626){3}{\usebox{\plotpoint}}
\multiput(1314,357)(20.191,-4.807){3}{\usebox{\plotpoint}}
\multiput(1377,342)(20.436,3.626){3}{\usebox{\plotpoint}}
\put(1439,353){\usebox{\plotpoint}}
\sbox{\plotpoint}{\rule[-0.400pt]{0.800pt}{0.800pt}}%
\put(191,457){\usebox{\plotpoint}}
\multiput(192.41,457.00)(0.502,0.556){117}{\rule{0.121pt}{1.090pt}}
\multiput(189.34,457.00)(62.000,66.737){2}{\rule{0.800pt}{0.545pt}}
\multiput(253.00,527.41)(1.616,0.505){33}{\rule{2.720pt}{0.122pt}}
\multiput(253.00,524.34)(57.355,20.000){2}{\rule{1.360pt}{0.800pt}}
\multiput(316.00,544.09)(0.919,-0.503){61}{\rule{1.659pt}{0.121pt}}
\multiput(316.00,544.34)(58.557,-34.000){2}{\rule{0.829pt}{0.800pt}}
\multiput(378.00,510.09)(2.192,-0.508){23}{\rule{3.560pt}{0.122pt}}
\multiput(378.00,510.34)(55.611,-15.000){2}{\rule{1.780pt}{0.800pt}}
\multiput(441.00,498.41)(1.210,0.504){45}{\rule{2.108pt}{0.121pt}}
\multiput(441.00,495.34)(57.625,26.000){2}{\rule{1.054pt}{0.800pt}}
\multiput(503.00,521.09)(2.323,-0.509){21}{\rule{3.743pt}{0.123pt}}
\multiput(503.00,521.34)(54.232,-14.000){2}{\rule{1.871pt}{0.800pt}}
\multiput(565.00,510.41)(0.934,0.503){61}{\rule{1.682pt}{0.121pt}}
\multiput(565.00,507.34)(59.508,34.000){2}{\rule{0.841pt}{0.800pt}}
\multiput(628.00,544.41)(0.706,0.502){81}{\rule{1.327pt}{0.121pt}}
\multiput(628.00,541.34)(59.245,44.000){2}{\rule{0.664pt}{0.800pt}}
\multiput(690.00,585.09)(1.026,-0.503){55}{\rule{1.826pt}{0.121pt}}
\multiput(690.00,585.34)(59.210,-31.000){2}{\rule{0.913pt}{0.800pt}}
\multiput(753.00,557.41)(0.646,0.502){89}{\rule{1.233pt}{0.121pt}}
\multiput(753.00,554.34)(59.440,48.000){2}{\rule{0.617pt}{0.800pt}}
\multiput(815.00,605.41)(1.511,0.505){35}{\rule{2.562pt}{0.122pt}}
\multiput(815.00,602.34)(56.683,21.000){2}{\rule{1.281pt}{0.800pt}}
\multiput(877.00,623.09)(2.361,-0.509){21}{\rule{3.800pt}{0.123pt}}
\multiput(877.00,623.34)(55.113,-14.000){2}{\rule{1.900pt}{0.800pt}}
\multiput(940.00,609.06)(9.995,-0.560){3}{\rule{10.120pt}{0.135pt}}
\multiput(940.00,609.34)(40.995,-5.000){2}{\rule{5.060pt}{0.800pt}}
\multiput(1002.00,607.41)(0.934,0.503){61}{\rule{1.682pt}{0.121pt}}
\multiput(1002.00,604.34)(59.508,34.000){2}{\rule{0.841pt}{0.800pt}}
\multiput(1065.00,638.09)(1.776,-0.506){29}{\rule{2.956pt}{0.122pt}}
\multiput(1065.00,638.34)(55.866,-18.000){2}{\rule{1.478pt}{0.800pt}}
\multiput(1127.00,623.41)(0.866,0.503){65}{\rule{1.578pt}{0.121pt}}
\multiput(1127.00,620.34)(58.725,36.000){2}{\rule{0.789pt}{0.800pt}}
\put(1189,658.34){\rule{12.800pt}{0.800pt}}
\multiput(1189.00,656.34)(36.433,4.000){2}{\rule{6.400pt}{0.800pt}}
\multiput(1252.00,663.41)(0.507,0.502){115}{\rule{1.013pt}{0.121pt}}
\multiput(1252.00,660.34)(59.897,61.000){2}{\rule{0.507pt}{0.800pt}}
\multiput(1314.00,721.09)(0.881,-0.503){65}{\rule{1.600pt}{0.121pt}}
\multiput(1314.00,721.34)(59.679,-36.000){2}{\rule{0.800pt}{0.800pt}}
\multiput(1377.00,688.41)(0.706,0.502){81}{\rule{1.327pt}{0.121pt}}
\multiput(1377.00,685.34)(59.245,44.000){2}{\rule{0.664pt}{0.800pt}}
\sbox{\plotpoint}{\rule[-0.500pt]{1.000pt}{1.000pt}}%
\put(191,457){\usebox{\plotpoint}}
\multiput(191,457)(18.564,9.282){4}{\usebox{\plotpoint}}
\multiput(253,488)(17.773,10.720){3}{\usebox{\plotpoint}}
\multiput(316,526)(14.442,14.907){5}{\usebox{\plotpoint}}
\multiput(378,590)(19.783,-6.280){3}{\usebox{\plotpoint}}
\multiput(441,570)(17.949,10.422){3}{\usebox{\plotpoint}}
\multiput(503,606)(18.199,-9.980){4}{\usebox{\plotpoint}}
\multiput(565,572)(19.957,5.702){3}{\usebox{\plotpoint}}
\multiput(628,590)(20.097,5.186){3}{\usebox{\plotpoint}}
\multiput(690,606)(13.455,15.804){5}{\usebox{\plotpoint}}
\multiput(753,680)(18.444,-9.519){3}{\usebox{\plotpoint}}
\multiput(815,648)(11.178,-17.488){6}{\usebox{\plotpoint}}
\multiput(877,551)(18.854,8.679){3}{\usebox{\plotpoint}}
\multiput(940,580)(20.314,-4.259){3}{\usebox{\plotpoint}}
\multiput(1002,567)(19.871,-5.993){3}{\usebox{\plotpoint}}
\multiput(1065,548)(13.984,15.337){5}{\usebox{\plotpoint}}
\multiput(1127,616)(20.540,2.982){3}{\usebox{\plotpoint}}
\multiput(1189,625)(17.897,10.511){3}{\usebox{\plotpoint}}
\multiput(1252,662)(20.436,3.626){3}{\usebox{\plotpoint}}
\multiput(1314,673)(20.756,0.000){3}{\usebox{\plotpoint}}
\multiput(1377,673)(19.460,-7.219){3}{\usebox{\plotpoint}}
\put(1439,650){\usebox{\plotpoint}}
\sbox{\plotpoint}{\rule[-0.600pt]{1.200pt}{1.200pt}}%
\put(191,457){\usebox{\plotpoint}}
\multiput(191.00,459.24)(0.907,0.500){58}{\rule{2.488pt}{0.121pt}}
\multiput(191.00,454.51)(56.836,34.000){2}{\rule{1.244pt}{1.200pt}}
\multiput(255.24,491.00)(0.500,0.718){116}{\rule{0.120pt}{2.033pt}}
\multiput(250.51,491.00)(63.000,86.780){2}{\rule{1.200pt}{1.017pt}}
\put(316,582.01){\rule{14.936pt}{1.200pt}}
\multiput(316.00,579.51)(31.000,5.000){2}{\rule{7.468pt}{1.200pt}}
\multiput(378.00,589.24)(4.505,0.503){6}{\rule{9.750pt}{0.121pt}}
\multiput(378.00,584.51)(42.763,8.000){2}{\rule{4.875pt}{1.200pt}}
\multiput(441.00,597.24)(1.147,0.500){44}{\rule{3.056pt}{0.121pt}}
\multiput(441.00,592.51)(55.658,27.000){2}{\rule{1.528pt}{1.200pt}}
\multiput(503.00,624.24)(1.105,0.500){46}{\rule{2.957pt}{0.121pt}}
\multiput(503.00,619.51)(55.862,28.000){2}{\rule{1.479pt}{1.200pt}}
\put(565,649.51){\rule{15.177pt}{1.200pt}}
\multiput(565.00,647.51)(31.500,4.000){2}{\rule{7.588pt}{1.200pt}}
\multiput(628.00,656.24)(2.276,0.501){18}{\rule{5.614pt}{0.121pt}}
\multiput(628.00,651.51)(50.347,14.000){2}{\rule{2.807pt}{1.200pt}}
\multiput(690.00,670.24)(3.841,0.502){8}{\rule{8.700pt}{0.121pt}}
\multiput(690.00,665.51)(44.943,9.000){2}{\rule{4.350pt}{1.200pt}}
\multiput(753.00,679.24)(2.114,0.501){20}{\rule{5.260pt}{0.121pt}}
\multiput(753.00,674.51)(51.083,15.000){2}{\rule{2.630pt}{1.200pt}}
\multiput(815.00,694.24)(0.510,0.500){110}{\rule{1.540pt}{0.120pt}}
\multiput(815.00,689.51)(58.804,60.000){2}{\rule{0.770pt}{1.200pt}}
\put(877,751.01){\rule{15.177pt}{1.200pt}}
\multiput(877.00,749.51)(31.500,3.000){2}{\rule{7.588pt}{1.200pt}}
\multiput(940.00,752.26)(3.777,-0.502){8}{\rule{8.567pt}{0.121pt}}
\multiput(940.00,752.51)(44.219,-9.000){2}{\rule{4.283pt}{1.200pt}}
\put(1002,742.51){\rule{15.177pt}{1.200pt}}
\multiput(1002.00,743.51)(31.500,-2.000){2}{\rule{7.588pt}{1.200pt}}
\multiput(1065.00,741.26)(3.319,-0.502){10}{\rule{7.740pt}{0.121pt}}
\multiput(1065.00,741.51)(45.935,-10.000){2}{\rule{3.870pt}{1.200pt}}
\put(1127,734.01){\rule{14.936pt}{1.200pt}}
\multiput(1127.00,731.51)(31.000,5.000){2}{\rule{7.468pt}{1.200pt}}
\multiput(1189.00,736.26)(3.375,-0.502){10}{\rule{7.860pt}{0.121pt}}
\multiput(1189.00,736.51)(46.686,-10.000){2}{\rule{3.930pt}{1.200pt}}
\multiput(1252.00,726.26)(0.732,-0.500){74}{\rule{2.071pt}{0.121pt}}
\multiput(1252.00,726.51)(57.701,-42.000){2}{\rule{1.036pt}{1.200pt}}
\multiput(1314.00,689.24)(1.123,0.500){46}{\rule{3.000pt}{0.121pt}}
\multiput(1314.00,684.51)(56.773,28.000){2}{\rule{1.500pt}{1.200pt}}
\multiput(1377.00,717.24)(1.563,0.501){30}{\rule{4.020pt}{0.121pt}}
\multiput(1377.00,712.51)(53.656,20.000){2}{\rule{2.010pt}{1.200pt}}
\sbox{\plotpoint}{\rule[-0.200pt]{0.400pt}{0.400pt}}%
\put(191.0,131.0){\rule[-0.200pt]{0.400pt}{155.380pt}}
\put(191.0,131.0){\rule[-0.200pt]{300.643pt}{0.400pt}}
\put(1439.0,131.0){\rule[-0.200pt]{0.400pt}{155.380pt}}
\put(191.0,776.0){\rule[-0.200pt]{300.643pt}{0.400pt}}
\end{picture}

\end{center}
\end{figure}

With a larger population, heterozygosity wanders, but much more slowly than it
did with 10 individuals, as expected.

Simulating 20 individuals for 1 million generations took 58.768 seconds on my
desktop, using a 2.8 GHz processor, including the output formatting.

\lstinputlisting[language=Python]{wright_fisher.py}

\end{document}
